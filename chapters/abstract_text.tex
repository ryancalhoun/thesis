\hspace{15pt} Examination of the behavior of Generalized Covariance Union (GCU) reveals a previously unsuspected superficial
relationship to the problem of finding a minimal enclosing ellipsoid of ellipsoids, also called a L\"owner ellipsoid. By
interpreting any mean and covariance pair of some estimate or measurement as the 1-$\sigma$ contour ellipsoid of its
associated Gaussian probability density function, the results of GCU appear to form a L\"owner 1-$\sigma$ ellipsoid
about the 1-$\sigma$ ellipsoids of its $n$ inputs. This thesis presents a means to analyze and test this behavior
numerically, detailing the one- and two-dimensional cases, using mathematics easily extensible into higher dimensions.
The current hypothesis, supported by experimental evidence, is that the relationship between GCU and the minimal
enclosing ellipsoid problem is one of equivalence. Subsequent to this finding, pending a formal proof, it will be
possible to apply tools from computational geometry to solve data fusion problems.

