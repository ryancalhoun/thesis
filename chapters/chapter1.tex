\chapter{Introduction}\label{chapter:introduction}

\section{Background Information}

\PARstart{O}{ver} the last decade, Level-1 information management has matured significantly, with the development of
rigorous algorithms robust to the effects of unmodeled correlations, and corrupt and/or spurious information, in the
context of general distributed data fusion networks. Recent inroads into the development of higher level information
management applications have yielded significant improvements to the data fusion tool called Covariance Union
(CU)~\cite{uhlmann03,julier05}. CU is used to find the union of a set of input modes or state estimates in a way that
guarantees an accurate representation of the system's true state in circumstances when other tools may not; simply put,
it is a method to reduce many modes to one.

A higher level data fusion application is likely manage information in a multimodal scheme, using either Multi
Hypothesis Tracking (MHT) or a Gaussian Mixture Model (GMM). MHT maintains multiple location and uncertainly estimates
corresponding to distinct possible states~\cite{barshalom88}. GMM is a parametrization of the Probability Density
Function (PDF) that describes the uncertainty distribution across the application. Since PDF approximations typically
only represent the significant modes of the distribution in terms of their mean and covariance, the representation
differs from that of MHT only in that a GMM interprets each mode as an explicit Gaussian PDF~\cite{fusion06}. Since
the implementation of these two techniques is identical, this thesis will use the GMM convention, of the form
  \begin{equation}\label{eqn:gmm}
    p(\mathbf{x})=\sum_{i=1}^Np_i\gaussN{\mathbf{x};\mu{_i},\sigma_i\sigma_i^T},
  \end{equation}
whenever referring to a multimodal state estimate, with the understanding that a Gaussian MHT model will be equivalent.

The fusion of a set $S$ of location and uncertainty estimates---each defining a possible state of the target, only one
of which is guaranteed to be consistent---with another set $T$ can be accomplished simply by forming the Cartesian
product $S\times T$ and applying the appropriate fusion algorithm (Kalman or Covariance Intersection) to the pairs.
Unfortunately, this yields a combined estimate that has $O(|S|*|T|)$ modes, which implies that the complexity of the
fused estimate exceeds that of the original estimates. This increasing complexity will tend to exhaust available
resources and therefore must be mitigated. CU can be used to combine modes, instead of pruning those deemed unnecessary
by some metric, and therefore reduce the information complexity of the fused set in a way that correctly estimates the
cumulative effect of the remaining modes~\cite{fusion06}.

\section{Notation}

Determining how to represent information and uncertainty, in either a single-target application or a multimodal system,
is a key first step that impacts all aspects of the data fusion problem. The representation must provide both an
estimate of the state of the target or system of interest {\em and} its associated degree of error or uncertainty, and
the uncertainty must be defined in a form that permits it to be empirically determined. There must be a rigorous
algorithm for fusing information in the representation, and the computational complexity of the representation and its
associated fusion algorithms must be bounded for practical application~\cite{fusion06}.

By far the most widely used information representation is the mean and covariance form, where the mean vector defines
the best estimate of the state of the target and the error covariance provides an upper bound on the expected squared
error associated with the mean. For example, the measured position of an object in two dimensions can be represented as
a vector $\v{m}$ consisting of the object's estimated mean position, e.g., $\v{m} = [x,y]^T$, and an
error covariance matrix $\m{M}$ that expresses the uncertainty associated with the estimated mean.  If the error in
the estimated mean vector is denoted as $\ea$, then the error covariance matrix is an estimate of the expected squared
error, E$[\ea\ea^T]$. The estimate is said to be {\em consistent} (or conservative) if and only if $\m{M}$ $\geq$
E$[\ea\ea^T]$ or, equivalently, $\m{M}$ - E$[\ea\ea^T]$ is positive definite or semidefinite (i.e., has no negative
eigenvalues). The full estimate of a target's state is given by the mean and covariance pair
$(\v{m},\m{M})$~\cite{fusion06,uhlmann03}. Although the GMM convention explicitly defines its own notation, as in
equation \ref{eqn:gmm}, the mean and covariance notation will be used throughout this thesis whenever referring to any
probabilistic estimate.


\section{Problem Definition}

Generalized Covariance Union (GCU) forms an important part of the hierarchy of data fusion tools which includes the
Kalman filter, Covariance Intersection, and the optimal form of CU. Each of these tools is capable of generating a
consistent result over a set of $n$ input estimates $(\v{m}_i,\m{M}_i)$, given certain preconditions. However, practical
running times for the current implementation of CU and GCU prohibit the use of these tools in any real-time
applications (operational cost is roughly proportional to the number of estimates $n$, and the cube of the
dimensionality of the state space)~\cite{fusion06}.

Research into GCU has revealed a remarkable superficial similarity to the problem of finding a Minimal Enclosing
Ellipsoid of Ellipsoids (MEEE), a common technique from computational geometry and computer graphics. These two problems
are very different, in the sense that mean and covariance estimates in tracking and data fusion refer to moments of a
possibly unknown underlying probability distribution, whereas ellipsoidal intersection and containment involves finite
volumes of space. Despite any differences between GCU and MEEE in representation and formulation, the similarities
between the results are suggestive of some level of equivalence.

This thesis will examine the GCU optimization problem and the possible applicability of techniques from computational
geometry, by presenting evidence that GCU and MEEE techniques may in fact be used to solve the same problem.


\section{Organization of Thesis}


The organization of this thesis is outlined as follows. Chapter \ref{chapter:introduction} begins by providing background
information and motivation for the research proposed in this thesis. A definition of the problem is given next, to
outline the scope of the research.

Work leading up to the research presented here is detailed in Chapter \ref{chapter:gcu}. This chapter will present the
Kalman filter, Covariance Intersection, and Covariance Union as part of a framework of data fusion tools. Generalized
Covariance Union is presented here as well, after this framework has been established.

Chapter \ref{chapter:meee} will define the problem of finding a Minimal Enclosing Ellipsoid of Ellipsoids, and give a
method to test whether a proposed enclosing ellipsoid properly contains its enclosed ellipsoids, as it should. Also given
in this chapter is a transformation between mean-covariance space and ellipsoid space, in order to provide a way for the
GCU and MEEE results to be compared.

Chapter \ref{chapter:analysis} will present the results of this research, experimental evidence and mathematical
analysis that demonstrate the relationship between GCU and MEEE.

This thesis concludes with a summary of accomplishments and dialog for future work in Chapter
\ref{chapter:conclusion}.



