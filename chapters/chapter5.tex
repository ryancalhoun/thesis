\chapter{Conclusion}\label{chapter:conclusion}

\section{Discussion}

\PARstart{T}{his thesis} has presented a technique from data fusion called Generalized Covariance Union, an important
member in the hierarchy of data fusion tools. GCU allows data fusion operations to be performed in the presence of
possibly correlated and/or spurious input, but its current implementation does not provide a real-time response in
higher-dimensional state spaces.

GCU can be graphically plotted in such a way as to suggest equivalence with the Minimal Enclosing Ellipsoid of
Ellipsoids problem, which would allow computational geometry tools to be applied to solve the GCU optimization problem.
The 1-$\sigma$ contour ellipsoids which uniquely define the Gaussian PDF associated with an error covariance, and the
covariance-to-ellipsoid transformation given in equation \ref{eqn:cu2e}, are interchangeable, and this provides a
mechanism for testing whether Generalized CU and MEEE do in fact solve the same problem. The logic outlined in section
\ref{section:explanation} and the experimental evidence illustrated in examples \ref{ex:gcu1d} and \ref{ex:gcu2d}, and
in section \ref{section:examples}, suggest that the answer to this question is yes.


\section{Future Work}

Although the experimental evidence presented is suggestive, it is not conclusive. Before research on GCU and MEEE can be
merged, there must be a rigorous proof in place showing that these two techniques are equivalent in any dimensionality.
Once this proof is established, it will be possible to solve GCU using MEEE algorithms, and vice-versa. The immediate
benefit will be a direct comparison of software running time of implementations of each technique, leading toward
improved real-time responses in both domains. Additionally, once it becomes accepted that GCU and MEEE are
interchangeable, this will allow for further investigation into other relationships between the fields of data fusion
and computational geometry. Perhaps these relationships exist not only between the two techniques used in this thesis,
but are in fact an indication of a larger degree of compatibility between two previously unrelated classes of problems.

This thesis contains the first formal analysis of GCU, following its initial presentation to the 9th International
Conference on Information Fusion in 2006, in~\cite{fusion06}. Planned future work includes a journal article covering
the GCU--MEEE equivalence. Although past and current research has been hampered by the lack of available tools---e.g. the
current implementation of CU/GCU is the first of its kind---continuing refinement to the CU/GCU optimization problem
will no doubt prove invaluable to further study, eventually leading to a robust real-time tool for mainstream data fusion
applications.


